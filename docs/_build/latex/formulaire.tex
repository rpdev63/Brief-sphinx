%% Generated by Sphinx.
\def\sphinxdocclass{report}
\documentclass[letterpaper,10pt,french]{sphinxmanual}
\ifdefined\pdfpxdimen
   \let\sphinxpxdimen\pdfpxdimen\else\newdimen\sphinxpxdimen
\fi \sphinxpxdimen=.75bp\relax
\ifdefined\pdfimageresolution
    \pdfimageresolution= \numexpr \dimexpr1in\relax/\sphinxpxdimen\relax
\fi
%% let collapsible pdf bookmarks panel have high depth per default
\PassOptionsToPackage{bookmarksdepth=5}{hyperref}

\PassOptionsToPackage{warn}{textcomp}
\usepackage[utf8]{inputenc}
\ifdefined\DeclareUnicodeCharacter
% support both utf8 and utf8x syntaxes
  \ifdefined\DeclareUnicodeCharacterAsOptional
    \def\sphinxDUC#1{\DeclareUnicodeCharacter{"#1}}
  \else
    \let\sphinxDUC\DeclareUnicodeCharacter
  \fi
  \sphinxDUC{00A0}{\nobreakspace}
  \sphinxDUC{2500}{\sphinxunichar{2500}}
  \sphinxDUC{2502}{\sphinxunichar{2502}}
  \sphinxDUC{2514}{\sphinxunichar{2514}}
  \sphinxDUC{251C}{\sphinxunichar{251C}}
  \sphinxDUC{2572}{\textbackslash}
\fi
\usepackage{cmap}
\usepackage[T1]{fontenc}
\usepackage{amsmath,amssymb,amstext}
\usepackage{babel}



\usepackage{tgtermes}
\usepackage{tgheros}
\renewcommand{\ttdefault}{txtt}



\usepackage[Sonny]{fncychap}
\ChNameVar{\Large\normalfont\sffamily}
\ChTitleVar{\Large\normalfont\sffamily}
\usepackage{sphinx}

\fvset{fontsize=auto}
\usepackage{geometry}


% Include hyperref last.
\usepackage{hyperref}
% Fix anchor placement for figures with captions.
\usepackage{hypcap}% it must be loaded after hyperref.
% Set up styles of URL: it should be placed after hyperref.
\urlstyle{same}

\addto\captionsfrench{\renewcommand{\contentsname}{Intro:}}

\usepackage{sphinxmessages}
\setcounter{tocdepth}{1}



\title{Formulaire}
\date{janv. 06, 2023}
\release{1.0.0}
\author{Sirine}
\newcommand{\sphinxlogo}{\vbox{}}
\renewcommand{\releasename}{Version}
\makeindex
\begin{document}

\ifdefined\shorthandoff
  \ifnum\catcode`\=\string=\active\shorthandoff{=}\fi
  \ifnum\catcode`\"=\active\shorthandoff{"}\fi
\fi

\pagestyle{empty}
\sphinxmaketitle
\pagestyle{plain}
\sphinxtableofcontents
\pagestyle{normal}
\phantomsection\label{\detokenize{index::doc}}



\chapter{Introduction}
\label{\detokenize{introduction:introduction}}\label{\detokenize{introduction::doc}}
\sphinxAtStartPar
This is simple sphinx documentation example.

\sphinxAtStartPar
Here, Explain the project objective and features:
\begin{itemize}
\item {} 
\sphinxAtStartPar
Ce code a pour but de créer une interface graphique (avec la bibliothèque tkinter) permettant de gérer une liste d’employés. L’interface comporte un formulaire permettant d’ajouter un employé à la liste en entrant son nom, prénom, poste et en cliquant sur le bouton « Enregistrer ». La liste est enregistrée dans un fichier CSV (Comma Separated Values) nommé « liste\_employes.csv ».

\end{itemize}

\sphinxAtStartPar
La fonction « check\_form » vérifie que les champs « Prénom » et « Nom » du formulaire ne sont pas vides, et affiche une erreur le cas échéant. Si les champs sont remplis, la fonction appelle la fonction « write\_line\_in\_csv » pour ajouter les données du nouvel employé au fichier CSV, puis affiche un message de validation.
La fonction « write\_line\_in\_csv » lit le fichier CSV, ajoute les données du nouvel employé à la fin de celui\sphinxhyphen{}ci et réécrit le fichier CSV avec les données mises à jour.
L’interface comporte également un listing des employés enregistrés dans le fichier CSV, qui est affiché sous le formulaire.


\section{HEADING:}
\label{\detokenize{introduction:heading}}\begin{itemize}
\item {} 
\sphinxAtStartPar
Formulaire vive les collegues;

\end{itemize}


\section{REFERENCES:}
\label{\detokenize{introduction:references}}\begin{itemize}
\item {} 
\sphinxAtStartPar
\sphinxhref{https://www.google.com}{Google} , search engine has been used throughout the project.

\item {} 
\sphinxAtStartPar
Other kind of text \sphinxcode{\sphinxupquote{Bold reference}}.

\item {} 
\sphinxAtStartPar
Bold \sphinxstylestrong{letters}.

\end{itemize}


\section{Author:}
\label{\detokenize{introduction:author}}
\sphinxAtStartPar
Sirine Mnassri Yousfi.


\chapter{vivelescollegues\_sirine}
\label{\detokenize{modules:vivelescollegues-sirine}}\label{\detokenize{modules::doc}}

\section{formulaire\_commenté module}
\label{\detokenize{formulaire_comment_xe9:module-formulaire_commente}}\label{\detokenize{formulaire_comment_xe9:module-formulaire_comment_xe9}}\label{\detokenize{formulaire_comment_xe9:formulaire-commente-module}}\label{\detokenize{formulaire_comment_xe9::doc}}\index{module@\spxentry{module}!formulaire\_commenté@\spxentry{formulaire\_commenté}}\index{formulaire\_commenté@\spxentry{formulaire\_commenté}!module@\spxentry{module}}
\sphinxAtStartPar
Module docstring
\index{check\_form() (dans le module formulaire\_commenté)@\spxentry{check\_form()}\spxextra{dans le module formulaire\_commenté}}

\begin{fulllineitems}
\phantomsection\label{\detokenize{formulaire_comment_xe9:formulaire_commente.check_form}}\pysiglinewithargsret{\sphinxcode{\sphinxupquote{formulaire\_commenté.}}\sphinxbfcode{\sphinxupquote{check\_form}}}{\emph{\DUrole{n}{first\_name}}, \emph{\DUrole{n}{name}}, \emph{\DUrole{n}{poste}}}{}
\sphinxAtStartPar
Cette fonction vérifie si les champs « Prénom » et « Nom » du formulaire d’ajout d’un employé sont remplis, et affiche un message d’erreur le cas échéant.
Si les champs sont remplis, elle ajoute le nouvel employé à la liste des employés et affiche un message de validation.
\begin{quote}
\begin{description}
\item[{:first\_name}] \leavevmode{[}str{]}
\sphinxAtStartPar
Le prénom de l’employé.

\item[{:name}] \leavevmode{[}str{]}
\sphinxAtStartPar
Le nom de l’employé.

\item[{:poste}] \leavevmode{[}str{]}
\sphinxAtStartPar
Le poste occupé par l’employé.

\end{description}
\end{quote}
\begin{quote}
\begin{description}
\item[{None}] \leavevmode
\sphinxAtStartPar
Cette fonction ne retourne rien, mais affiche des messages d’erreur ou de validation et modifie le fichier CSV « liste\_employes.csv » si nécessaire.

\end{description}
\end{quote}

\begin{sphinxVerbatim}[commandchars=\\\{\}]
\PYG{g+gp}{\PYGZgt{}\PYGZgt{}\PYGZgt{} }\PYG{n}{check\PYGZus{}form}\PYG{p}{(}\PYG{l+s+s1}{\PYGZsq{}}\PYG{l+s+s1}{Alice}\PYG{l+s+s1}{\PYGZsq{}}\PYG{p}{,} \PYG{l+s+s1}{\PYGZsq{}}\PYG{l+s+s1}{\PYGZsq{}}\PYG{p}{,} \PYG{l+s+s1}{\PYGZsq{}}\PYG{l+s+s1}{Ingénieur}\PYG{l+s+s1}{\PYGZsq{}}\PYG{p}{)}
\PYG{g+go}{\PYGZsh{} Un message d\PYGZsq{}erreur est affiché indiquant que le nom est manquant.}
\PYG{g+gp}{\PYGZgt{}\PYGZgt{}\PYGZgt{} }\PYG{n}{check\PYGZus{}form}\PYG{p}{(}\PYG{l+s+s1}{\PYGZsq{}}\PYG{l+s+s1}{\PYGZsq{}}\PYG{p}{,} \PYG{l+s+s1}{\PYGZsq{}}\PYG{l+s+s1}{Dupont}\PYG{l+s+s1}{\PYGZsq{}}\PYG{p}{,} \PYG{l+s+s1}{\PYGZsq{}}\PYG{l+s+s1}{Ingénieur}\PYG{l+s+s1}{\PYGZsq{}}\PYG{p}{)}
\PYG{g+go}{\PYGZsh{} Un message d\PYGZsq{}erreur est affiché indiquant que le prénom est manquant.}
\PYG{g+gp}{\PYGZgt{}\PYGZgt{}\PYGZgt{} }\PYG{n}{check\PYGZus{}form}\PYG{p}{(}\PYG{l+s+s1}{\PYGZsq{}}\PYG{l+s+s1}{Alice}\PYG{l+s+s1}{\PYGZsq{}}\PYG{p}{,} \PYG{l+s+s1}{\PYGZsq{}}\PYG{l+s+s1}{Dupont}\PYG{l+s+s1}{\PYGZsq{}}\PYG{p}{,} \PYG{l+s+s1}{\PYGZsq{}}\PYG{l+s+s1}{Ingénieur}\PYG{l+s+s1}{\PYGZsq{}}\PYG{p}{)}
\PYG{g+go}{\PYGZsh{} Le nouvel employé Alice Dupont est ajouté à la liste des employés et un message de validation est affiché.}
\end{sphinxVerbatim}

\end{fulllineitems}

\index{write\_line\_in\_csv() (dans le module formulaire\_commenté)@\spxentry{write\_line\_in\_csv()}\spxextra{dans le module formulaire\_commenté}}

\begin{fulllineitems}
\phantomsection\label{\detokenize{formulaire_comment_xe9:formulaire_commente.write_line_in_csv}}\pysiglinewithargsret{\sphinxcode{\sphinxupquote{formulaire\_commenté.}}\sphinxbfcode{\sphinxupquote{write\_line\_in\_csv}}}{\emph{\DUrole{n}{first\_name}}, \emph{\DUrole{n}{name}}, \emph{\DUrole{n}{date}}, \emph{\DUrole{n}{poste}}}{}
\sphinxAtStartPar
Cette fonction ajoute une nouvelle ligne au fichier CSV contenant la liste des employés.
La nouvelle ligne est créée à partir des informations fournies en entrée : nom, prénom, date d’embauche et poste.
\begin{quote}
\begin{description}
\item[{:first\_name}] \leavevmode{[}str{]}
\sphinxAtStartPar
Le prénom de l’employé.

\item[{:name}] \leavevmode{[}str{]}
\sphinxAtStartPar
Le nom de l’employé.

\item[{:date}] \leavevmode{[}str{]}
\sphinxAtStartPar
La date d’embauche de l’employé, au format « YYYY\sphinxhyphen{}MM\sphinxhyphen{}DD ».

\item[{:poste}] \leavevmode{[}str{]}
\sphinxAtStartPar
Le poste occupé par l’employé.

\end{description}
\end{quote}
\begin{quote}
\begin{description}
\item[{None}] \leavevmode
\sphinxAtStartPar
Cette fonction ne retourne rien, mais modifie le fichier CSV « liste\_employes.csv » en ajoutant une nouvelle ligne.

\end{description}
\end{quote}

\begin{sphinxVerbatim}[commandchars=\\\{\}]
\PYG{g+gp}{\PYGZgt{}\PYGZgt{}\PYGZgt{} }\PYG{n}{write\PYGZus{}line\PYGZus{}in\PYGZus{}csv}\PYG{p}{(}\PYG{l+s+s1}{\PYGZsq{}}\PYG{l+s+s1}{Alice}\PYG{l+s+s1}{\PYGZsq{}}\PYG{p}{,} \PYG{l+s+s1}{\PYGZsq{}}\PYG{l+s+s1}{Dupont}\PYG{l+s+s1}{\PYGZsq{}}\PYG{p}{,} \PYG{l+s+s1}{\PYGZsq{}}\PYG{l+s+s1}{2022\PYGZhy{}01\PYGZhy{}01}\PYG{l+s+s1}{\PYGZsq{}}\PYG{p}{,} \PYG{l+s+s1}{\PYGZsq{}}\PYG{l+s+s1}{Ingénieur}\PYG{l+s+s1}{\PYGZsq{}}\PYG{p}{)}
\PYG{g+go}{\PYGZsh{} Le fichier CSV \PYGZdq{}liste\PYGZus{}employes.csv\PYGZdq{} est modifié et contient maintenant une nouvelle ligne avec les informations sur l\PYGZsq{}employé Alice Dupont.}
\end{sphinxVerbatim}

\end{fulllineitems}



\chapter{Indices and tables}
\label{\detokenize{index:indices-and-tables}}\begin{itemize}
\item {} 
\sphinxAtStartPar
\DUrole{xref,std,std-ref}{genindex}

\item {} 
\sphinxAtStartPar
\DUrole{xref,std,std-ref}{modindex}

\item {} 
\sphinxAtStartPar
\DUrole{xref,std,std-ref}{search}

\end{itemize}


\renewcommand{\indexname}{Index des modules Python}
\begin{sphinxtheindex}
\let\bigletter\sphinxstyleindexlettergroup
\bigletter{f}
\item\relax\sphinxstyleindexentry{formulaire\_commenté}\sphinxstyleindexpageref{formulaire_commenté:\detokenize{module-formulaire_commente}}
\end{sphinxtheindex}

\renewcommand{\indexname}{Index}
\printindex
\end{document}